\documentclass[12pt,a4paper]{article}
\usepackage[utf8]{inputenc}
\usepackage[magyar]{babel}
\usepackage[T1]{fontenc}
\usepackage{amsmath}
\usepackage{amsfonts}
\usepackage{amssymb}
\usepackage{listings}
\usepackage{color}
\usepackage{textcomp}
\definecolor{listinggray}{gray}{0.9}
\definecolor{lbcolor}{rgb}{0.9,0.9,0.9}
\lstset{
	%backgroundcolor=\color{lbcolor},
	tabsize=4,
	rulecolor=,
	language=perl,
        basicstyle=\scriptsize,
        upquote=true,
        aboveskip={1.5\baselineskip},
        columns=fixed,
        showstringspaces=false,
        extendedchars=true,
        breaklines=true,
        prebreak = \raisebox{0ex}[0ex][0ex]{\ensuremath{\hookleftarrow}},
        showtabs=false,
        showspaces=false,
        showstringspaces=false,
        identifierstyle=\ttfamily,
        keywordstyle=\color[rgb]{0,0,1},
        commentstyle=\color[rgb]{0.133,0.545,0.133},
        stringstyle=\color[rgb]{0.627,0.126,0.941},
}

\title{UNIX rendszer felhasználói és fejlesztői felülete\\
HTTP szerver Perl nyelven}
\author{Guba Sándor}
\date{\today}
\begin{document}
\maketitle
\newpage
\section{Részletes specifikáció}
A szkript alkalmas az egyszerűbb HTTP (Hypertext Transfer Protocol) kérések feldolgozására. Képes feldolgozni a GET, POST és HEAD kéréseket. Védelmet nyujt a root könyvtárból való kitörés ellen mint a "http://host/../../etc/passwd". Alkalmas perl CGI szkriptek futtatására. Logolja a hibás kimenetelű lekérdezéseket (404 Not Found). A beállítások egyszerűen változtathatóak a konfigurációs fájlban. A konfigurációban állítható a root könyvtár (DocumentRoot), a cgi (EnableCgi) futtatás engedélyezése, index fájlok nevei (IndexFiles), a logolás (LogFile) helye valamint egyedi 404 oldal (DefaultErrorPage). A program használható inetd szerver segítségével vagy a mellékelt httpstandalone.perl program segítségével.

\section{Működési vázlat}
Elindításkor a program inicializálja a változókat, valamint betölti a konfigurációs állományt. Ha ez sikertelen hibával megáll. A program a standard inputon várakozik bemenetre. A bemenetet soronként dolgozza fel. Ha egy bemenetre illeszkedik egy reguláris kifejezés pl.: /\^{}GET/ akkor az ehhez tartozó utasításokat végrehajtja. A fejléc mező végén /\^{}\textbackslash r/ elkezdi a kérés tényleges feldolgozását. Felépítjük a HTTP szabványnak megfelelő fejléc részt amit a standard kimenetre írunk. A megfelelő fájl tartalmát a standard kimenetre írjuk, vagy futtatjuk. Amennyiben egy fájl nem elérhető úgy hibakódot adunk vissza. A programot használhatjuk inetd szerverrel valamint a mellékelt httpstandalone.perl programmal is. Utóbbi tulajdonképpen az inetd feladatait látja el. Indításkor a megfelelő paramétereket feldolgozza és a megadott porton hallgat. Új kapcsolat esetén fork() segítségével új példányt hozunk létre. Az open3() függvény segítségével elindítjuk a megadott http feldolgozót. Ezt követően a hálózati bemenetet a parser bemenetére irányítjuk, valamint a parser kimenetét a hálózati adatfolyamba. A gyerek program ezt követően megáll.
\section{Belső szerkezet}

\paragraph{Inicializáció} A program elején inicializáljuk a változókat, feldolgozzuk a konfigurációs fájlt.
\begin{lstlisting}
#!/usr/bin/perl 
use File::Copy;
use IPC::Open3;
#Disable Buffering!
my $old_fh = select(STDOUT);
$| = 1;
select($old_fh);
$old_fh = select(STDIN);
$| = 1;
select($old_fh);

#Determines wheater read/execute the request
#
# 0 - Read
# 1 - GET
# 2 - POST
#
$readorexecute = 0;
#
#Variable for the full route for the queried file
$get_file = "";
#
#Indicates the end of the HEADER part
$payload = 0;
#
#Associative array for settings
%settings;

#
#Load setting from file (First command line parameter)
%settings=&LoadSettings($ARGV[0]);
#
#Setting up variables listed in configuration file
$baseaddress = $settings{'DocumentRoot'};
chomp($baseaddress);

$Default_Error_Page=$settings{'DefaultErrorPage'};
chomp($Default_Error_Page);
$Enable_Cgi=$settings{'EnableCgi'};
chomp($Enable_Cgi);
@Index_Files=split(/,/ , $settings{'IndexFiles'} );
$Log_File=$settings{'LogFile'};  
chomp($Log_File);

#
#Logging
$LOGLINE = "";
my($logpath) = $Log_File."/"."ErrorLog.log";
chomp($logpath);
open(ERROR_LOG, ">>", "$logpath") or die("Invalid Path for LogFile: $Log_File");
\end{lstlisting}
\paragraph{Fő ciklus} Itt történik a standard bementről való olvasás és értelmezés. A  különböző sorok különböző reguláris kifejezésre illeszkednek és bneállítják a hozzájuk tartozó változókat.

\begin{lstlisting}
while( <STDIN> )
{
    #Get '../' pattern out of the line! Preventing from ../../../etc/passwd etc...
    s/\.\.\//\//g;
    #Split lines at spaces
    @data = split(/ /, $_);
    #Parsing the input
    /^HEAD/ && do {
	$get_file = $baseaddress . $data[1];
	chomp($get_file);
	&HTTP_HEAD($get_file);
	last;
    };
    /^GET/ && do {
	
	$LOGLINE = $data[0]." ".$data[1];
	#Temporary created route
        my($testing_path) = $baseaddress . $data[1];
	chomp($testing_path);
	#Check if root folder
	if( $data[1] eq "/")
	{
	    $get_file=&GetFolderIndex($data[1]);
	    if( $get_file =~ m/\.cgi/  and $Enable_Cgi eq "1" )
	    {
		$readorexecute = 1;
	    }
	}
	#Check for directory
	elsif( -d $testing_path )
	{
	    $get_file = &GetFolderIndex($data[1]);
            if( $get_file =~ m/\.cgi/  and $Enable_Cgi eq "1" )
            {
                $readorexecute = 1;
            }
	}
	#Check for executable
        elsif( $data[1] =~ m/\.cgi/ and ($Enable_Cgi eq "1") )
        {
	    #Check for GET paramaters
            my(@parsed_get) = split(/\?/, $data[1]);
            my($filename) = $parsed_get[0];
            $get_params = $parsed_get[1];
            #Set environment variables
            $ENV{"REQUEST_METHOD"} = 'GET';
            $ENV{'QUERY_STRING'} = $get_params;
            #Execute script
            $get_file = $baseaddress . $filename;
            $readorexecute = 1;
        }
        else
        {
	    #If not Directory/Executable it's a sample html or txt
            $get_file = $baseaddress . $data[1];
        }
    };
    #POST method for CGI
    /^POST/ && do {
	if($Enable_Cgi eq "1")
	{
            $get_file = $baseaddress . $data[1];
            $ENV{"REQUEST_METHOD"} = 'POST';
            $readorexecute = 2;
	}
        else
        {
            #If not Directory/Executable it's a sample html or txt
            $get_file = $baseaddress . $data[1];
        }
    };
    #Set the Content-Length
    /^Content-Length:/ && do {
        $ENV{"CONTENT_LENGTH"}=$data[1];
    };
    #End of HEADER section
    /^\r?$/ && do {
	#Remove \n etc from file route
        chomp($get_file);
        #Execute POST CGI
        if( $readorexecute == 2)
        {
	    #Read (Content_Length) DATA from <STDIN>
	    my($params);
	    read(STDIN, $params, $ENV{'CONTENT_LENGTH'});
	    #Execute Script passing parameters
	    &FileExecuter($get_file,$params);
	    last;
        }
	#Execute GET CGI
        elsif( $readorexecute == 1){
            &FileExecuter($get_file,"");
            last;
        }		
	#Read file
        else{
            &FileReader($get_file);
            last;
        }
    };
}
\end{lstlisting}

\paragraph{Funkciók}
\subparagraph{HTTP\_HEAD} Ha a lekérdezés létező fájlra vonatkozik, akkor felépítjük a válasz HEADER részét.
\begin{lstlisting}
sub HTTP_HEAD{
my ($path) = @_;
if( -e $path )
{
    #HTTP_OK
    &HTTP_HEAD_OK;
    #HTTP_SERVER_HEAD
    &HTTP_SERVER_H;
    #Content-type:
    &HTTP_CONTENT_T($path);
    #Header end
    print("\r\n");
}
else{
    &HTTP_ERROR_404($path);
}

}
\end{lstlisting}

\subparagraph{FileExecuter} CGI script esetén ez a függvény felel a szkript lefuttatásáért.
\begin{lstlisting}
sub FileExecuter{
    my ($path,$params) = @_;
    if( -e $path )                                                                             
    {   
	#HTTP 200
        &HTTP_HEAD_OK;
	#
        #Fork and Exec with I/O pipe
        $pid=open3(\*CHILD_IN,\*CHILD_OUT,\*CHILD_ERR,"$path");
	#On POST copy data to CHILD_STDIN
        if( $readorexecute > "1" )
	{
	     print CHILD_IN "$params";
	}
	close(CHILD_IN);
	#Forward the CGIs output
        while(<CHILD_OUT>)                                                                     
        {   
            print($_);                                                                         
        }
        waitpid($pid, 0);
        close(CHILD_OUT);                                                                      
        close(CHILD_ERR);
	close(STDIN);
	&safe_exit;                                                                                 
    }
    else                                                                                       
    {   
        &HTTP_ERROR_404($path);                                                                        
    } 
}
\end{lstlisting}

\subparagraph{FileReader} Hagyományos html/text vagy egyéb bináris esetén ez a funkció másolja ki a fájlt a standard outputra.
\begin{lstlisting}
sub FileReader{
    my ($path) = @_;
    my ($exist) = 1;
    open(FILE, $path) or $exist = 0;
    if( $exist > 0)
    {
        #HTTP_OK
        &HTTP_HEAD_OK;
	#HTTP_SERVER_HEAD
	&HTTP_SERVER_H;
        #Content-type:
        &HTTP_CONTENT_T($path);
        #Header end
        print("\r\n");
        #Copy function to push data on stdout
        copy($path,\*STDOUT);
        close(FILE);
    }
    else
    {
        &HTTP_ERROR_404($path);
    }
    &safe_exit;
}
\end{lstlisting}

\subparagraph{HTTP\_HEAD\_OK} A fejrész első sora pozitiív választ ad lekérdezésre.
\begin{lstlisting}
sub HTTP_HEAD_OK{
    print("HTTP/1.1 200 OK\n");
    &HTTP_SERVER_H;
}
\end{lstlisting}

\subparagraph{HTTP\_SERVER\_H} Szerver specifikus fejrészek: verzió, dátum.
\begin{lstlisting}
sub HTTP_SERVER_H{
    print("Date: ");
    system("date");
    print("Server: Perl Test Server (TFDAZ6)\r\n");
}
\end{lstlisting}

\subparagraph{HTTP\_CONTENT\_T} Ez a függvény felelős a fájlok típusának meghatározására.
\begin{lstlisting}
sub HTTP_CONTENT_T{
    my ($get_file)=@_;
    print("Content-type: ");
    system("/usr/bin/file -bi $get_file");
}
\end{lstlisting}

\subparagraph{LoadSettings} Ez a funkció értelmezi a paraméterként kapott konfigurációs állományt.
\begin{lstlisting}
sub LoadSettings{
    my($cfgfile) = @_;
    open(SETTINGS, "<$cfgfile");
    my (%settings);
    while(<SETTINGS>)
    {
        /^\#/ && do { next;};
        /^[a-zA-Z]/ && do {
            my(@parsed) = split(/ /,$_);
            $settings{$parsed[0]}=$parsed[1];
        };
    }
    close(SETTINGS);
    return %settings;
}
\end{lstlisting}

\subparagraph{GetFolderIndex} Megkeresi a mappához tartozó index állományt. Ehhez a konfigurációban megadott neveket használja fel.
\begin{lstlisting}
sub GetFolderIndex{
    my($path)=@_;
    my($indexed) = "";
    #Check for the / at the end of the path
    if( $path =~ m/\/$/ )
    {
	$indexed = $baseaddress.$path;
    }
    else
    {
	$indexed = $baseaddress.$path."/";
    }
    chomp($indexed);
    foreach(@Index_Files)
    {
        my($temp) = $_;
	chomp($temp);
	if( -f $indexed.$temp)
	{
	    return $indexed.$temp;
	}
    }
    &HTTP_ERROR_404;
}
\end{lstlisting}

\subparagraph{HTTP\_ERROR\_404} Hiba esetén a 404-es errort lekezelő függvény.
\begin{lstlisting}
sub HTTP_ERROR_404{
    my( $path) = @_;
    &ErrorLog("404 Not Found");
    my ($exist) = 1;
    my ($path) = $baseaddress . "/" . $Default_Error_Page;
    chomp($path);
    open(FILE, $path) or $exist = 0;
    if( $exist > 0)
    {
	&HTTP_ERR_HEAD;
	&HTTP_CONTENT_T($path);
	print("\r\n");

	copy($path,\*STDOUT);
	close(FILE);
    }
    else
    {
	&HTTP_ERR_HEAD;
        &HTTP_ERR_BODY;
    }
    &safe_exit;
}
sub HTTP_ERR_HEAD{
    print("HTTP/1.1 404 FILE NOT FOUND\r\n");
    &HTTP_SERVER_H;
}
sub HTTP_ERR_BODY{
    print("Content-type: text/plain; charset=UTF-8\r\n");
    print("\r\n");                                                                             
    print("A kert lap nem jelenitheto meg... \n");                                             
    print("\r\n"); 
}
\end{lstlisting}

\subparagraph{ErrorLog} Ez a funkció írja szükséges esetben a logfájlt.
\begin{lstlisting}
sub ErrorLog()
{
    my($msg)=@_;
    $date = `date`;
    chomp($date);
    chomp($LOGLINE);
    print ERROR_LOG "[$date] ERROR: $msg | $LOGLINE\n";
}
\end{lstlisting}

\subparagraph{safe\_exit} Kilépés esetén a lefoglalt erőforrásokat felszabadítja.
\begin{lstlisting}
sub safe_exit{
    close(ERROR_LOG);
    exit;
}
\end{lstlisting}

\section{Tesztelés}
A tesztelés kézzel generált http kérésekkel történt. A programot elindítva a standard bemenetre manuálisan beírt sorokra a standard kimeneten olvasható a válasz. A tesztelést előre rögzített http kérésekkel (http\_log) gyorsítottam. A különböző funkciók teszteléshez szükséges volt továbbá egy minta környezet kialakítása. Ez a mellékletben a www mappa. Ebben megtalálható mélyebb mappa struktúra, GET és POST műveleteket feldolgozó CGI szkriptek, különböző nevű index fájlok. 
\section{Használat}
A használati útmutató a httpserver.perl.man.8 man dokumentumban olvasható.
\paragraph{Konfigurációs fájl}
Példa a konfigurációs fájlra:
\lstinputlisting{httpserver.cfg}

\newpage
\section{Fájllista}
\begin{table}[h]
\centering
\begin{tabular}{ll}
\hline
Fájl neve               & Leírás \\
\hline
\noindent
httpserver.perl & A HTTP parser.\\
httpserver.cfg & A HTTP parserhez tartozó konfigurációs fájl. \\
httpstandalone.perl & Hálózati kiszolgáló a HTTP parserhez.\\
httpserver.perl.man.8 & A programhoz tartozó felhasználói leírás. \\
www/index.html & Tesztállomány. \\
www/404.html & Tesztállomány egyedi hibaoldalhoz. \\
www/img.png & Tesztállomány bináris kérésekhez. \\
www/post.html & Tesztállomány POST kérés teszteléséhez. \\
www/post.cgi & Tesztállomány POST kérés teszteléséhez. \\
www/get.cgi & Tesztállomány GET kérés teszteléséhez. \\
www/date.cgi & Tesztállomány CGI futtatásának teszteléséhez. \\
www/public/index.htm & Tesztállomány egyedi index fájl nevekhez. \\
www/cgi/index.cgi & Tesztállomány index fájlokhoz, CGI futtatással.\\
\hline
\end{tabular}
\end{table}

\newpage
\tableofcontents

\end{document}